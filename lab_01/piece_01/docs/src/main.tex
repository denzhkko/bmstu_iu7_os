%% Преамбула TeX-файла

% 1. Стиль и язык
\documentclass[utf8x]{G7-32} % Стиль (по умолчанию будет 14pt)
\usepackage[T2A]{fontenc}
\usepackage[russian]{babel}
% Остальные стандартные настройки убраны в preamble.inc.tex.
\include{preamble.inc}

% Настройки листингов.
\include{listings.inc}

% Полезные макросы листингов.
\include{macros.inc}

%For titul
%--------------------------------------
\usepackage{graphicx}
\graphicspath{ {./images/} }
\usepackage{tabularx} % in the preamble
\usepackage[normalem]{ulem}
%--------------------------------------

\begin{document}

\begin{titlepage}
    \thispagestyle{empty}

    \noindent\begin{minipage}{0.05\textwidth}
        \includegraphics[scale=0.3]{bmstu}
    \end{minipage}
    \hfill
    \begin{minipage}{0.85\textwidth}\raggedleft
        \begin{center}
            \fontsize{10pt}{0.3\baselineskip}\selectfont \textbf{Министерство науки и высшего образования Российской Федерации \\ Федеральное государственное бюджетное образовательное учреждение \\ высшего образования \\ <<Московский государственный технический университет \\ имени Н.Э. Баумана \\ (национальный исследовательский университет)>> \\ (МГТУ им. Н.Э. Баумана)}
        \end{center}
    \end{minipage}

    \begin{center}
        \fontsize{12pt}{0.1\baselineskip}\selectfont
        \noindent\makebox[\linewidth]{\rule{\textwidth}{4pt}} \makebox[\linewidth]{\rule{\textwidth}{1pt}}
    \end{center}

    \begin{flushleft}
        \fontsize{12pt}{0.8\baselineskip}\selectfont

        ФАКУЛЬТЕТ \uline{
            \hfill
            Информатика и системы управления
            \hfill}

        КАФЕДРА \uline{\mbox{\hspace{4mm}}
            \hfill
            Программное обеспечение ЭВМ и информационные технологии
            \hfill}
    \end{flushleft}

    \vfill
    
    \begin{center}
        \fontsize{19pt}{\baselineskip}\selectfont

        \textbf{Отчет по лабораторной работе №1} \\
        \textbf{по теме "Дисассемблирование INT 8h"}
    \end{center}

    \vfill
    
    \begin{tabularx}{\textwidth}{Xcc}
        Студент \uline{Недолужко Д.В.} \\
        Группа \uline{ИУ7-53Б} \\
        Преподаватель \uline{Рязанова Н.Ю.} \\
    \end{tabularx}
    
    \vfill
    
    \begin{center}
        \normalsize Москва \\
        \the\year ~г.
    \end{center}
\end{titlepage}

\frontmatter % выключает нумерацию ВСЕГО; здесь начинаются ненумерованные главы: реферат, введение, глоссарий, сокращения и прочее.

% Команды \breakingbeforechapters и \nonbreakingbeforechapters
% управляют разрывом страницы перед главами.
% По-умолчанию страница разрывается.

% \nobreakingbeforechapters
% \breakingbeforechapters


\mainmatter % это включает нумерацию глав и секций в документе ниже

\chapter{Ассемблерный код}

  \lstinputlisting[
    language={[x86masm]Assembler},
    caption=Обработчик INT 8h,
    basicstyle=\scriptsize
  ]{./listings/int8h.asm}
  
  \lstinputlisting[
    language={[x86masm]Assembler},
    caption=Сопрограмма sub\_1,
    basicstyle=\scriptsize
  ]{./listings/sub_1.asm}

\chapter{Схемы алгоритмов}

  \section{Схема алгоритма обработчика INT 8h}
  
    \begin{center}
      \includegraphics[scale=0.6]{images/scheme-int8h_1.drawio.png}
    \end{center}
    
    \begin{center}
      \includegraphics[scale=0.6]{images/scheme-int8h_2.drawio.png}
    \end{center}
    
    \begin{center}
      \includegraphics[scale=0.6]{images/scheme-int8h_3.drawio.png}
    \end{center}
  
  \section{Схема алгоритма сопрограммы sub\_01}

    \begin{center}
      \includegraphics[scale=0.6]{images/scheme-sub_1.drawio.png}
    \end{center}


\end{document}

%%% Local Variables:
%%% mode: latex
%%% TeX-master: t
%%% End:
